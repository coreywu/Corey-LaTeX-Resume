\documentclass[11pt, letterpaper, oneside]{article}

\usepackage{geometry}
\usepackage[T1]{fontenc}
\usepackage[dvipsnames]{color}
\usepackage{baskervald}
\usepackage{multirow}
\usepackage{tabularx}
\usepackage{multicol}
\usepackage{enumitem}

%-----------------------------------------------------------
%Margin setup

\setlength{\voffset}{0.1in}
\setlength{\paperwidth}{8.5in}
\setlength{\paperheight}{11in}
\setlength{\headheight}{0in}
\setlength{\headsep}{0in}
\setlength{\textheight}{11in}
\setlength{\textheight}{9.5in}
\setlength{\topmargin}{-0.25in}
\setlength{\textwidth}{7in}
\setlength{\topskip}{0in}
\setlength{\oddsidemargin}{-0.25in}
\setlength{\evensidemargin}{-0.25in}

\pagestyle{empty}

% Name and contact information
\newcommand{\name}{Corey Wu}
\newcommand{\program}{2A Computer Engineering}
\newcommand{\website}{http://www.coreywu.com}
\newcommand{\phone}{(647) 502-9912}
\newcommand{\email}{c69wu@uwaterloo.ca}

% A horizontal rule for sections
\newcommand{\HRule}[2]{\textcolor{#1}{\rule{\linewidth}{#2}}}
 
% A section tile command
\newcommand{\sectiontitle}[1]{\begin{minipage}{\textwidth}\vspace{-10pt}\begin{flushleft}\hspace{-20pt}\vspace{-25pt}
\Large\MakeUppercase{#1}\end{flushleft}\end{minipage}\\\HRule{black}{0.15mm}\vspace{\baselineskip}}

% An environment for a resume section
\newenvironment{ressection}[1]{
  \sectiontitle{#1}}
  {\vspace{-\baselineskip}}
  
% Resentry defines a position
% Arg1: Position title
% Arg2: Company or Office---Company
% Arg3: Location, PROV
% Arg4: Date Range
\newcommand{\resentryheader}[4]{
    \vspace{-5pt}
    \textbf{#1}\hspace{\stretch{1}}\textcolor{black}{#3}\\
    \textit{#2}\hspace{\stretch{1}}\textcolor{black}{#4}\\
}
 
% A resitem is a simple list element
\newcommand{\resitem}[1]{
    \vspace{2pt}
    \item \begin{flushleft} #1 \end{flushleft}
}

\newcommand{\resinneritem}[1]{
	\vspace{-5pt}
    \item \begin{flushleft} #1 \end{flushleft}
}

% This is a resume entry with a header and bullet items
\newenvironment{resentry}[4]{
  \begin{minipage}{\textwidth}
    \resentryheader{#1}{#2}{#3}{#4}
        \vspace{-\baselineskip}
    \begin{itemize}[noitemsep,nolistsep]
}{
    \end{itemize}
        \vspace{\baselineskip}
        \end{minipage}
}

\begin{document}

\begin{ressection}{Experience}
  \begin{resentry}{Test Automation Developer}{Polar Inc. (formerly Polar Mobile)}{Toronto, ON}{Apr. 2013 -- Aug. 2013}
    \resitem{Designed automated testing architecture using Buildbot to automatically run tests written in Java.}
    \resitem{Helped company achieve continuous deployment by setting continuous integration using Buildbot, reducing cycle time significantly.}
    \vspace{4pt} \hspace{-15pt}
    \textbf{Key Technologies:} Java, JUnit, Eclipse, Selenium, Buildbot.
  \end{resentry}
  
\end{ressection}

\begin{ressection}{Projects}
  \begin{resentry}{UW Food Menu}{Waterloo Food Services Android App}{}{Mar. 2013 -- Present}
    \resitem{Designed a parser to asynchronously retrieve and parse JSON data from the Waterloo API into data to be displayed in app.}
    \resitem{Integrated with Google Maps API to display location data.}
    \resitem{Created intuitive UI using popular Android design patterns.}
    \resitem{Implemented Python and PHP scripts to parse API data and store in a MySQL or SQLite database.}
    \vspace{4pt} \hspace{-15pt}
    \textbf{Key Technologies:} Java, Eclipse, Python, PHP, MySQL.
  \end{resentry}
\end{ressection}

\begin{ressection}{Education}
  \begin{resentry}{Candidate for Bachelor of Applied Science}{University of Waterloo)}{Waterloo, ON}{Sept. 2012 -- Present}
    \resitem{\textbf{Relevant Projects}}
	\textbf{Android Map Navigation App}   
    \begin{itemize}
		\resinneritem{Designed an Android application to algorithmically navigate a user past obstacles to a chosen destination using an SVG map.}
		\resinneritem{Implemented constant polling from magnetic field and gyroscope sensors to determine the user's position and orientation.}
		\resinneritem{Created a finite state machine to determine user's movements.}
		
	\end{itemize}
  \end{resentry}
\end{ressection}

\end{document}